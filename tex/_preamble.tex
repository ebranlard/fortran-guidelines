\usepackage[T1]{fontenc}
% macro to select a scaled-down version of Bera Mono (for instance)
% \makeatletter
% \newcommand\BeraMonottfamily{%
%   \def\fvm@Scale{0.9}% scales the font down
%   \fontfamily{fvm}\selectfont% selects the Bera Mono font
% }
% \makeatother

% \usepackage{DejaVuSansMono}
% \usepackage{dejavu}
%% which loads the DejaVu Serif and DejaVu Sans fonts as well
% \renewcommand*\familydefault{\ttdefault}
\usepackage{lmodern}
\usepackage{amsmath}
\usepackage{amssymb}
\usepackage{listings} % a inclure pour la fonction listing
\usepackage{longtable}
\usepackage{geometry}
\usepackage{ifpdf}
\ifpdf
	\usepackage{graphicx}
	\usepackage{epstopdf} %don't forget the shell-escape flag for pdflatex
\else
	\usepackage{graphicx}
\fi
\usepackage{hyperref}
\usepackage{color} 
\definecolor{mygreen}{rgb}{0,0.6,0}
\definecolor{gret}{rgb}{0,0.6,0}
\definecolor{mygray}{rgb}{0.5,0.5,0.5}
\definecolor{mymauve}{rgb}{0.58,0,0.82}
\definecolor{orange}{rgb}{1,0.5,0}
\definecolor{grey}{rgb}{0.88,0.88,0.88}
\definecolor{midgrey}{rgb}{0.5,0.5,0.50}
\definecolor{greylight}{rgb}{0.95,0.95,0.95}
\definecolor{greensoft}{rgb}{0.25,0.95,0.25}
%%for dtu
\definecolor{dtured}{rgb}{0.6706,0.20784,0.227451}
\definecolor{dtugrey}{rgb}{0.5,0.5,0.5}

\geometry{
a4paper,
body={175mm,265mm},
left=17mm,
top=15mm,
headheight=7mm,
headsep=4mm,
footskip=4mm,
marginparsep=4mm,
marginparwidth=27mm}
% ---SPACING TWEAKS
\usepackage{titlesec}
% \titlespacing\section{0pt}{12pt plus 4pt minus 2pt}{0pt plus 2pt minus 2pt}
% \titlespacing\subsection{0pt}{0pt}{-0pt}
% \setlength{\aboveitemizeskip}{-05pt}
% \setlength{\belowitemizeskip}{-3pt}
% \setlength{\superparsep}{0.2cm}
% \setlength{\captiontabsep}{-0.2cm}
%\linespread{1.5} % line spacing
% ---TOC tweaks
%\setlength{\cftbeforechapskip}{2ex}
%\setlength{\cftbeforepartskip}{1mm}
% \setlength{\cftbeforesecskip}{0.05cm}
% \setlength{\cftsecindent}{1cm}
% \renewcommand{\cftsecfont}{}
% \renewcommand{\cftsecdotsep}{\cftdotsep}
%\setlength{\cftbeforesubsecskip}{-6pt}
% --- Maths
\setlength{\abovedisplayskip}{0.05cm} %space before maths/equations
\setlength{\belowdisplayskip}{0.05cm} %space after maths
% --- PARAGRAPHS
\setlength{\parskip}{0.1cm}
\setlength{\parindent}{0in}

% --------------------------------------------------------------------------------
% --- Code 
% --------------------------------------------------------------------------------
\newcommand{\basiclstset}{%
\lstset{ %
  backgroundcolor=\color{white},   % choose the background color; 
  basicstyle=\ttfamily\footnotesize,        % the size of the fonts that are used for the code
%   basicstyle=\BeraMonottfamily, 
  breakatwhitespace=false,         % sets if automatic breaks should only happen at whitespace
  breaklines=true,                 % sets automatic line breaking
  commentstyle=\color{mygreen},    % comment style
  deletekeywords={...},            % if you want to delete keywords from the given language
  escapeinside={\%*}{*)},          % if you want to add LaTeX within your code
  extendedchars=true,              % lets you use non-ASCII characters; for 8-bits encodings only, does not work with UTF-8
  frame=single,	                   % adds a frame around the code
  keepspaces=true,                 % keeps spaces in text, useful for keeping indentation of code (possibly needs columns=flexible)
  keywordstyle=\color{blue},       % keyword style
  language=Fortran,                 % the language of the code
  otherkeywords={defined},            % if you want to add more keywords to the set
%   numbersep=5pt,                   % how far the line-numbers are from the code
%   numberstyle=\tiny\color{mygray}, % the style that is used for the line-numbers
  rulecolor=\color{black},         % if not set, the frame-color may be changed on line-breaks within not-black text (e.g. comments (green here))
%   showspaces=false,                % show spaces everywhere adding particular underscores; it overrides 'showstringspaces'
%   showstringspaces=false,          % underline spaces within strings only
  showtabs=false,                  % show tabs within strings adding particular underscores
%   stepnumber=2,                    % the step between two line-numbers. If it's 1, each line will be numbered
  stringstyle=\color{mymauve},     % string literal style
%   tabsize=2,	                   % sets default tabsize to 2 spaces
%   title=\lstname,                   % show the filename of files included with \lstinputlisting; also try caption instead of title
%   belowskip=-2.0 \baselineskip,
%   belowcaptionskip=0cm,
%    abovecaptionskip=0cm,
%   aboveskip=0.1cm,
   belowskip=0em,
% linewidth=\linewidth
}
}
\basiclstset
% \lstset{
%   basicstyle=\BeraMonottfamily, 
%   frame=single,
% }
\newcommand{\cmd}[1]{\texttt{\detokenize{#1}}}

% --------------------------------------------------------------------------------
% --- Code related commands 
% --------------------------------------------------------------------------------
\lstloadlanguages{Fortran}

\lstnewenvironment{code}
{\null\hfill\minipage{0.48\textwidth}}
    {\endminipage\hfill\null}

\lstnewenvironment{codea}
{\hfill\minipage[b]{0.48\textwidth}}
    {\endminipage}
\lstnewenvironment{codeb}
{\hfill\minipage[b]{0.48\textwidth}}
  {\endminipage\hfill\ \newline\null}

\lstnewenvironment{codefull}
{\minipage{1.0\textwidth}}
    {\endminipage\newline\null}

\lstnewenvironment{good}
{\hfill\minipage{0.47\textwidth}\lstset{title=\textbf{good}}}
    {\endminipage\hfill\null}

\lstnewenvironment{bad}
{\hfill\minipage{0.47\textwidth}\lstset{title=\textbf{bad}}}
    {\endminipage\hfill\null}

    \newcommand{\includecode}[1]{\lstinputlisting[title=\texttt{\detokenize{#1}}]{#1}}
% --------------------------------------------------------------------------------
% --- User-defined commands
% --------------------------------------------------------------------------------
\newcommand{\weird}[1]{\textcolor{red}{!!!#1 }}

\newcommand{\topict}[1]{%
   \multicolumn{2}{p{\textwidth}}{\newline\textbf{#1}}\\
}
\newcommand{\reasont}[1]{%
    \multicolumn{2}{p{\textwidth}}{#1}\\
}
\newcommand{\topic}[1]{%
    ~\vspace{-0.2cm}\par\textbf{#1}\\
}
\newcommand{\reason}[1]{%
    #1\\
}
