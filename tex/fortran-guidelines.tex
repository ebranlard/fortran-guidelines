\documentclass{article} 
\usepackage[T1]{fontenc}
% macro to select a scaled-down version of Bera Mono (for instance)
% \makeatletter
% \newcommand\BeraMonottfamily{%
%   \def\fvm@Scale{0.9}% scales the font down
%   \fontfamily{fvm}\selectfont% selects the Bera Mono font
% }
% \makeatother

% \usepackage{DejaVuSansMono}
% \usepackage{dejavu}
%% which loads the DejaVu Serif and DejaVu Sans fonts as well
% \renewcommand*\familydefault{\ttdefault}
\usepackage{lmodern}
\usepackage{amsmath}
\usepackage{amssymb}
\usepackage{listings} % a inclure pour la fonction listing
\usepackage{longtable}
\usepackage{geometry}
\usepackage{ifpdf}
\ifpdf
	\usepackage{graphicx}
	\usepackage{epstopdf} %don't forget the shell-escape flag for pdflatex
\else
	\usepackage{graphicx}
\fi
\usepackage{hyperref}
\usepackage{color} 
\definecolor{mygreen}{rgb}{0,0.6,0}
\definecolor{gret}{rgb}{0,0.6,0}
\definecolor{mygray}{rgb}{0.5,0.5,0.5}
\definecolor{mymauve}{rgb}{0.58,0,0.82}
\definecolor{orange}{rgb}{1,0.5,0}
\definecolor{grey}{rgb}{0.88,0.88,0.88}
\definecolor{midgrey}{rgb}{0.5,0.5,0.50}
\definecolor{greylight}{rgb}{0.95,0.95,0.95}
\definecolor{greensoft}{rgb}{0.25,0.95,0.25}
%%for dtu
\definecolor{dtured}{rgb}{0.6706,0.20784,0.227451}
\definecolor{dtugrey}{rgb}{0.5,0.5,0.5}

\geometry{
a4paper,
body={175mm,265mm},
left=17mm,
top=15mm,
headheight=7mm,
headsep=4mm,
footskip=4mm,
marginparsep=4mm,
marginparwidth=27mm}
% ---SPACING TWEAKS
\usepackage{titlesec}
% \titlespacing\section{0pt}{12pt plus 4pt minus 2pt}{0pt plus 2pt minus 2pt}
% \titlespacing\subsection{0pt}{0pt}{-0pt}
% \setlength{\aboveitemizeskip}{-05pt}
% \setlength{\belowitemizeskip}{-3pt}
% \setlength{\superparsep}{0.2cm}
% \setlength{\captiontabsep}{-0.2cm}
%\linespread{1.5} % line spacing
% ---TOC tweaks
%\setlength{\cftbeforechapskip}{2ex}
%\setlength{\cftbeforepartskip}{1mm}
% \setlength{\cftbeforesecskip}{0.05cm}
% \setlength{\cftsecindent}{1cm}
% \renewcommand{\cftsecfont}{}
% \renewcommand{\cftsecdotsep}{\cftdotsep}
%\setlength{\cftbeforesubsecskip}{-6pt}
% --- Maths
\setlength{\abovedisplayskip}{0.05cm} %space before maths/equations
\setlength{\belowdisplayskip}{0.05cm} %space after maths
% --- PARAGRAPHS
\setlength{\parskip}{0.1cm}
\setlength{\parindent}{0in}

% --------------------------------------------------------------------------------
% --- Code 
% --------------------------------------------------------------------------------
\newcommand{\basiclstset}{%
\lstset{ %
  backgroundcolor=\color{white},   % choose the background color; 
  basicstyle=\ttfamily\footnotesize,        % the size of the fonts that are used for the code
%   basicstyle=\BeraMonottfamily, 
  breakatwhitespace=false,         % sets if automatic breaks should only happen at whitespace
  breaklines=true,                 % sets automatic line breaking
  commentstyle=\color{mygreen},    % comment style
  deletekeywords={...},            % if you want to delete keywords from the given language
  escapeinside={\%*}{*)},          % if you want to add LaTeX within your code
  extendedchars=true,              % lets you use non-ASCII characters; for 8-bits encodings only, does not work with UTF-8
  frame=single,	                   % adds a frame around the code
  keepspaces=true,                 % keeps spaces in text, useful for keeping indentation of code (possibly needs columns=flexible)
  keywordstyle=\color{blue},       % keyword style
  language=Fortran,                 % the language of the code
  otherkeywords={defined},            % if you want to add more keywords to the set
%   numbersep=5pt,                   % how far the line-numbers are from the code
%   numberstyle=\tiny\color{mygray}, % the style that is used for the line-numbers
  rulecolor=\color{black},         % if not set, the frame-color may be changed on line-breaks within not-black text (e.g. comments (green here))
%   showspaces=false,                % show spaces everywhere adding particular underscores; it overrides 'showstringspaces'
%   showstringspaces=false,          % underline spaces within strings only
  showtabs=false,                  % show tabs within strings adding particular underscores
%   stepnumber=2,                    % the step between two line-numbers. If it's 1, each line will be numbered
  stringstyle=\color{mymauve},     % string literal style
%   tabsize=2,	                   % sets default tabsize to 2 spaces
%   title=\lstname,                   % show the filename of files included with \lstinputlisting; also try caption instead of title
%   belowskip=-2.0 \baselineskip,
%   belowcaptionskip=0cm,
%    abovecaptionskip=0cm,
%   aboveskip=0.1cm,
   belowskip=0em,
% linewidth=\linewidth
}
}
\basiclstset
% \lstset{
%   basicstyle=\BeraMonottfamily, 
%   frame=single,
% }

% --------------------------------------------------------------------------------
% --- Code related commands 
% --------------------------------------------------------------------------------
\lstloadlanguages{Fortran}

\lstnewenvironment{code}
{\null\hfill\minipage{0.48\textwidth}}
    {\endminipage\hfill\null}

\lstnewenvironment{codea}
{\hfill\minipage[b]{0.48\textwidth}}
    {\endminipage}
\lstnewenvironment{codeb}
{\hfill\minipage[b]{0.48\textwidth}}
  {\endminipage\hfill\ \newline\null}

\lstnewenvironment{codefull}
{\minipage{1.0\textwidth}}
    {\endminipage\newline\null}

\lstnewenvironment{good}
{\hfill\minipage{0.47\textwidth}\lstset{title=\textbf{good}}}
    {\endminipage\hfill\null}

\lstnewenvironment{bad}
{\hfill\minipage{0.47\textwidth}\lstset{title=\textbf{bad}}}
    {\endminipage\hfill\null}

    \newcommand{\includecode}[1]{\lstinputlisting[title=\texttt{\detokenize{#1}}]{#1}}
% --------------------------------------------------------------------------------
% --- User-defined commands
% --------------------------------------------------------------------------------
\newcommand{\weird}[1]{\textcolor{red}{!!!#1 }}

\newcommand{\topict}[1]{%
   \multicolumn{2}{p{\textwidth}}{\newline\textbf{#1}}\\
}
\newcommand{\reasont}[1]{%
    \multicolumn{2}{p{\textwidth}}{#1}\\
}
\newcommand{\topic}[1]{%
    ~\vspace{-0.2cm}\par\textbf{#1}\\
}
\newcommand{\reason}[1]{%
    #1\\
}

\title{Some Fortran guidelines and pitfall}
% --------------------------------------------------------------------------------
% --- Document 
% --------------------------------------------------------------------------------
\begin{document}
\maketitle

\tableofcontents

\subsection*{Introduction}
\paragraph{Portability Portability Portability}
It is rather easy to write a code that is \textit{none standards, platform-dependent, architecture-dependent, compiler-dependent and library-dependent}.
Writting a portable code takes more effort on the programmer's level but it has its benefits: robustness, reusablility and more users. You learn a lot and find a lot of bugs simply by using different compilers.

\paragraph{Strategies}
There is probably two main strategies to ensure portability: (i) using preprocessor directives or (ii) wrapping non-portable code in separate files.
(i) Preprocessors directives are widely used, they are written close to the code which makes them easy to implement. The down side is they pollute the code and make it often very hard to read. Also, there is an inherent dependence on a specific preprocessor, and thus this reduces portability. The C-preprocessor is built-in in most compilers, and is probably the one that should be used if a preprocessor approach is chosen. 
(ii) Chosing to wrap non-portable code into separate files keeps the whole code perfectly readable. Further, the developpers responsible for a given library, architecture or OS have only a few files to work with, the rest of the code being ``universal''.
The down side is that it increases the number of source files and requires the ``Makefile'' or the ``Project'' to select the proper files. 
Also, not all the power of the preprocessor can be achieved in this way.
This is for instance the case for decoration of procedures or data like \texttt{bind(C)} or \texttt{DLLEXPORT} (see \autoref{sec:dlls}).
A combination of both approaches is likely to be the optimal solution.

\paragraph{In this document}
The guidelines written in this document are an attempt to lean towards portability. The content of the document might evolve in the future.
Wrapping of non-portable code into a separate module is the solution generally applied in this document. Different files are used to provide the differerent implementations of the interface of a given wrap-module.  Portability modules are called \verb|Support*| and their source code is located in the folder \cmd{_support*}.
For instance, the kind of a real is stored as \verb|MK| in the \verb|SupportPrecision| module (see \autoref{sec:precision}).

If recent standards offer options that helps portability, then this document will tend to favor these options despite the broken compatibility with regards to older compilers. This is the case for the \texttt{bind(C)} decoration which is very helpful for portability.

The following convention is used in this document:\\
\begin{codea}
! BAD PRACTICE
\end{codea}
\begin{codeb}
! GOOD PRACTICE
\end{codeb}


\paragraph{In this repository} Preliminary attempts of ``portable'' modules and makefiles are provided in this repository.  Most of the code originates from \cmd{Omnivor}'s implementation. It used to be compatible linux/windows and gfortran/intel/compaq/sun/portland, though it was not fully tested recently.
They are far from ``universal'' and will hopefully evolve in the future.

\section{Recommendations}


% --------------------------------------------------------------------------------
% --- Files
% --------------------------------------------------------------------------------
\subsection{Files and modules: towards a modular library-oriented code}
\topic{Files and modules: general guidelines}
\begin{enumerate}
    \item Non-portable code should be placed in a separate directory to avoid polluting the ``universal'' code.
    \item Modules containing derived types definitions should not contain anything else: no data or routines (though parameters/constants can be fine).  This seriously reduces the chances to run into circular dependencies problems. 
    \item Modules containing data should only contain this data: no types or routines. This reduces circular dependencies problems. This helps the identification of the data ``stored'' by the librairies. In fact, it's best to reduce data modules to a minimum (see next point).
    \item Data modules should be avoided. In a ``library''-like implementation, the ``user'' owns the data, not the library. The library is made of tools that manipulate the user's data but does not store anything. This makes the library thread safe and the code implemented can be more readily re-used.
    \item Use one module per files, it makes it easier to find them and reduces circular dependencies problem.
    \item The following structure can be adopted:
        \begin{itemize}
            \item \verb|AirfoilTypes.f90|  (contains no data)
            \item \verb|AirfoilParams.f90| (contains compile time constants, e.g. to increase code readibility)
            \item \verb|AirfoilTools.f90|  (contains no data, manipulate derive type instances as arguments)
            \item \verb|AirfoilData.f90| (fine, but the aim is to remove this module at the end, see point 4 above)
        \end{itemize}
    \item File extensions: Some OS are case-sensitive. Make sure you display file extensions in your file manager. It's good to stay consistent in the extension you use for your source files. It helps creating makefiles that can be used on any platforms. Some general conventions are:
        \begin{itemize}
            \item \verb|.f90|: Fortran 90 code
            \item \verb|.f|  : Old fortran code
            \item \verb|.F*|  : Fortran code that needs a preprocessor or Template for code generator (e.g. cheetah)
        \end{itemize}
    \item Avoid having two files with the same name in your project. It's easier to implement makefiles then.
\end{enumerate}

\topic{Modules: write implicit none in module only, it propagates to contained routines}
\begin{codea}
module SupportCompiler
  implicit none
contains
  subroutine foo()
    implicit none ! Not needed
  end subroutine
end module 
\end{codea}
\begin{codeb}
module SupportCompiler
  implicit none
contains
  subroutine foo()

  end subroutine
end module 
\end{codeb}\\

\topic{Modules: write explicit use-statements to avoid polluting your scope and help the reader}
\begin{codefull}
subroutine foo()
  use SupportCompiler, only: FORTRAN_COMPILER
  use AirfoilTools, only: airfoil_load, airfoil_interp
  use NewTools, only: init             ! using    the subroutine init from NewTools
  use OldTools, only: init_old => init ! renaming the subroutine init from OldTools
  ![...]
  call init()
  call init_old()
  ![...]
end subroutine 
\end{codefull}




% --------------------------------------------------------------------------------
% --- Variables init
% --------------------------------------------------------------------------------
\subsection{Declarations, initialization, allocations}
\topic{Initialization: not in definitions but straight after (except for derived types, see~\autoref{sec:derivedtypes})} 
\begin{codea}
subroutine foo()
  integer :: i = 0 !< implies save, bad!
  real(MK), pointer :: p=>null() !< idem
   

end subroutine 
\end{codea} 
\begin{codeb}
subroutine foo()
  integer :: i
  real(MK), pointer :: p
  i=0        ! safe
  nullify(p) ! safe
end subroutine 
\end{codeb}
\reason{Inializing in the definition implies save, it's very bad practice and can lead to disastrous surprises (see examples in \cmd{_unit_tests}). Save is in general a bad practice. Note: hawc2-intel is compiled with the -save flag by default! This has to change.} 
% --------------------------------------------------------------------------------
\topic{Arguments declaration: use intent in declarations, except for pointers}
\begin{codea}
function(x,y,i,p)
  ! Arguments
  real(MK) :: x !<
  real(MK) :: y !<
  integer  :: i !<
  integer, pointer :: p !<
  ! Variables
end
\end{codea} 
\begin{codeb}
function(x,y,i,p)
  ! Arguments
  real(MK), intent(in)   :: x !< best
  real(MK), intent(out)  :: y !< best
  integer, intent(inout) :: i !< best
  integer, pointer :: p
  ! Variables
end
\end{codeb}
\reason{More compiler optimizations can take place and errors detected at compilation time.}
% --------------------------------------------------------------------------------
\topic{Initialization: always initialize after allocation}
\begin{codea}
allocate(x(1:10));! x is garbage
\end{codea} 
\begin{codeb}
allocate(x(1:10)); x(1:10)= 0.0_MK ! safe
\end{codeb}
\reason{Compilers have flags to define the behavior of allocate (e.g. set to 0 or NaN). It's more portable not to rely on it. In statements and declarations, specifying the bounds explicitely is good practice: it reminds the reader of the dimensions, it helps the compiler, and bound mismatch can be found by the compiler.}
% --------------------------------------------------------------------------------
\topic{Allocations: the safe way}
\begin{codea}
allocate(x(1:10));
! Code above can crash with no backtrace
x(1:10)=0.0_MK


! 
! 
\end{codea} 
\begin{codeb}
allocate(x(1:10),stat=ierr); 
if (ierr/=0) print*,'x alloc error'; STOP
x(1:10)=0.0_MK

! or, using a wrapped function:
use MemoryManager, only: allocate_safe
call allocate_safe('x', x, 10, 0.0_MK)
\end{codeb}


\subsection{Stack pitfalls}
% --------------------------------------------------------------------------------
\topic{Stack: do not use assumed size local variable in routines}
\begin{codea}
function(x,n)
! Arguments
integer,intent(in) :: n
real(MK),dimension(n),intent(inout) :: x !< ok
! Variables
real(MK), dimension(n) :: y !< BAD!

end
\end{codea} 
\begin{codeb}
function(x,n)
! Arguments
integer,intent(in) :: n
real(MK),dimension(n),intent(inout) :: x !< ok
! Variables
real(MK), dimension(:), allocatable :: y !< OK
allocate(y(1:n)); y(1:n)= 0.0_MK         !  OK
end
\end{codeb}
\reason{The assumed size local variables are allocated on the stack and this might result in stack overflows or corruptions.}
% --------------------------------------------------------------------------------
\topic{Stack: do not use intrinsic functions for large arrays/vectors (they sometimes use the stack)}
\begin{codea}
maxval  ! Examples of known functions 
maxloc  ! acting on array/vector that can use
pack    ! the stack (e.g. Intel compiler)
\end{codea} 
\begin{codeb}
! Instead, write your own custom function
! using a for loop and no assumed size!
! See e.g. PackFunction in folder _tools
\end{codeb}
\reason{Segmentation faults can result from not following this guideline (if your stack is indeed too small).}
% --------------------------------------------------------------------------------
\topic{Stack: linux systems}
\begin{codea}
\end{codea} 
\begin{codeb}
ulimit -s unlimited
\end{codeb}
% \reason{Segmentation faults can result from not following this guideline.}
% --------------------------------------------------------------------------------


% --------------------------------------------------------------------------------
% --- If statements
% --------------------------------------------------------------------------------
\subsection{If-statements / comparisons}
\label{sec:if}
\topic{If-statements: logical comparison}
\begin{codea}
if(my_logical.eq.your_logical) print*,'bad'
if(my_logical.eq..true.) print*,'bad'
if(my_logical.eq..false.) print*,'bad'
\end{codea} 
\begin{codeb}
if(my_logical.eqv.your_logical.) print*,'good'
if(my_logical) print*,'good'
if(.not.my_logical) print*,'good'
\end{codeb}
\reason{For logical comparison \texttt{.eqv.} should be used. Most of the time, it can be omitted.}
% --------------------------------------------------------------------------------
\topic{If-statements: real equality comparison}
\begin{codea}
if(x == 12.0_MK) ! dangerous real comparison



!
\end{codea} 
\begin{codeb}
if(.not.(abs(x-12.0_MK)>0.0_MK))     ! ex1
!
if(precision_equal(x,12.0_MK))       ! ex1b
!
if(abs(x-12.0_MK)<precision(1.0_MK)) ! ex2
\end{codeb}
\reason{In the above, ex1 is allright but not so readable.  Ex1b uses a wrapped function that can be placed e.g. in the \texttt{SupportPrecision} module.
Ex2 is more ``physical'':  $x$ and $12$ are compared assuming the typical physical scale of the problem is $1$. Indeed, the machine precision for a given real value is different depending on the real kind (here MK) and the absolute value. }
% --------------------------------------------------------------------------------
\topic{If-statements: there is no assumed order of evaluation (example here for optional argument)}
\begin{codea}
if (present(x) .and. x>0) then 
   ! Compiler might evaluate x>0 first
   ! =>Segfault if x is not present
endif
\end{codea} 
\begin{codeb}
if (present(x)) then
  if (x>0) then ! safe
  endif
endif
\end{codeb}
% --------------------------------------------------------------------------------

% --------------------------------------------------------------------------------
% --- Do loops
% --------------------------------------------------------------------------------
\subsection{Do loops and memory order}
\label{sec:do}
\topic{Do loops: first index should runs the fastest to respect memory order}
\begin{codea}
do i=1,n
    do j=1,m ! bad, j run the fastest
        a(i,j)=1.0_MK
    enddo
enddo
\end{codea} 
\begin{codeb}
do j=1,m
    do i=1,n ! good
        a(i,j)=1.0_MK
    enddo
enddo
\end{codeb}
% --------------------------------------------------------------------------------
\topic{Memory: typical array dimensions for 3D geometry}
\begin{codea}
real(MK), dimension(n,3)  :: Points ! bad
\end{codea} 
\begin{codeb}
real(MK), dimension(3,n)  :: Points ! ok
\end{codeb}
% --------------------------------------------------------------------------------
\topic{Do loops: iteration on reals are bad practice}
\begin{codea}
do x=0._MK, 10._MK, 0.1_MK
enddo
\end{codea} 
\begin{codeb}
! Bad practice
! use loop on integer instead
\end{codeb}
% --------------------------------------------------------------------------------



% --------------------------------------------------------------------------------
% ---  Derived types
% --------------------------------------------------------------------------------
\subsection{Derived types}
\label{sec:derivedtypes}
\topic{Derived types:  use initializations, especially for pointers, always =>null() them}
\begin{codea}
type T_mytype
  real(MK), pointer :: p
  integer :: i
end type
\end{codea} 
\begin{codeb}
type T_mytype
  real(MK), pointer :: p=>null() !< always!
  integer :: i=0 !< safe to rely on it
end type
\end{codeb}
\reason{Components initialization is standard.} 
% --------------------------------------------------------------------------------
\topic{Derived types: component access}
\begin{codea}
T.i = 0 ! . is not standard
\end{codea} 
\begin{codeb}
T%i = 0  ! % is OK
\end{codeb}
% --------------------------------------------------------------------------------
\topic{Derived types: deallocate the components before the parent}
\begin{codea}
type T_mytype
  real(MK), pointer :: p=>null() 
end type
![...]
type(T_mytype), pointer :: t
nullify(t)
![...]
if (associated(t)) then 

   deallocate(t) ! potential memory loss
endif
\end{codea} 
\begin{codeb}
type T_mytype
  real(MK), pointer :: p=>null()
end type
![...]
type(T_mytype), pointer :: t
nullify(t)
![...]
if (associated(t)) then
    if (associated(t%p)) deallocate(t%p) 
    deallocate(t) ! fine
endif
\end{codeb}
% --------------------------------------------------------------------------------
\topic{Derived types: automatic code generation}
A bit of advertisment here, \cmd{simple-fortran-parser} can generate automatic code for derived types (like read/write to binary, init/dealloc). 



% --------------------------------------------------------------------------------
% ---  Characters
% --------------------------------------------------------------------------------
\subsection{Characters}
\label{sec:characters}
\topic{Characters: use the \texttt{len} specification}
\begin{codea}
character*16 :: s ! not standard
\end{codea} 
\begin{codeb}
character(len=16) :: s
\end{codeb}
% --------------------------------------------------------------------------------
\topic{Characters: use the \texttt{len} specification for arguments}
\begin{codea}
function f(s)
  character*(*) :: s ! akward character array
end
\end{codea} 
\begin{codeb}
function f(s)
  character(len=*), intent(in) :: s
end
\end{codeb}

% --------------------------------------------------------------------------------
\topic{Characters array: it's best to used fixed length for old compilers}
\begin{codefull}
character(len=20), dimension(5)              :: strings  ! fine
character(len=20), dimension(:), allocatable :: strings2 ! fine 
character(len=:),  dimension(:), allocatable :: strings3 ! fine but not for old compilers
strings(1)='a'
print*,string(1)(1:20)
! Allocation: 
! allocate(strings2(size(strings,1),size(strings,2)))                ! WRONG
allocate(character(len=len(strings(1))) :: strings2(size(strings,1)))! GOOD
\end{codefull}\par 
% --------------------------------------------------------------------------------
\topic{Characters: retrieving a string from C}
\begin{codefull}
subroutine string_switch(s_c) BIND(C, NAME='string_switch')
    use SupportPrecision, only: C_CHAR
    use CStrings,         only: cstring2fortran, fortranstring2c ! see folder _tools
    ! Argument
    character(kind=C_CHAR,len=1),dimension(*),intent(inout) :: s_c !< c string
    ! Variable
    character(len=255) :: s_f   !< fortran string
    ! [...]
    call cstring2fortran(s_c,s_f)
    ! [...]
    s_f='fortran'
    call fortranstring2c(s_f,s_c)
end 
\end{codefull}


% --------------------------------------------------------------------------------
% --- Arrays
% --------------------------------------------------------------------------------
\subsection{Arrays}
\label{sec:arrays}
\topic{Arrays: array construct with double dot is not standard}
\begin{codea}
integer, dimension(10) :: x

x=[1:10] ! not standard




!
\end{codea} 
\begin{codeb}
integer, dimension(10) :: x
integer :: i
x=[ (i,i=1,10) ] ! allright
x=(/(i,i=1,10)/) ! even more portable

do i=1,10
    x(i)=i ! readable, less bug
enddo
\end{codeb}
% --------------------------------------------------------------------------------


% --------------------------------------------------------------------------------
% --- File IO
% --------------------------------------------------------------------------------
\subsection{File IO}
\label{sec:fileio}
\topic{Unit value: don't use a fixed value}
\begin{codea}


open(99, ...) ! what if 99 is already opened?
read(99)
\end{codea} 
\begin{codeb}
use MainIO, only: get_free_unit()
iunit=get_free_unit()
open(iunit, ...)
read(iunit)
\end{codeb}
\reason{MainIO is defined in \texttt{\_tools}}
% --------------------------------------------------------------------------------
\topic{Binaries with direct access (e.g. Mann box): watch for the record length, wrap it in a module!}
\begin{codea}
!
open(iunit,file='u',recl=1,& !no standard 
    access='direct',form='unformatted',&
    status='old')
\end{codea} 
\begin{codeb}
use SupportCompiler, only: RECORD_LENGTH
open(iunit,file='u',recl=RECORD_LENGTH,& 
    access='direct',form='unformatted',&
    status='old')
\end{codeb}
\reason{Unfortunately, there is no standard for what recl should be. For intel and compaq by default \texttt{recl=1}. For gfortran (or intel with the flag \texttt{-assume byterecl}) \texttt{recl=4} }
% --------------------------------------------------------------------------------
\topic{Binaries with stream access (e.g. VTK bin): not available on old compilers}
\begin{codea}
open(iunit,'a.dat',form='UNFORMATTED',&
     access = 'stream', action = 'WRITE',&
     convert='BIG_ENDIAN') 
\end{codea} 
\begin{codeb}
use SupportCompiler, only: open_stream_write
call open_stream_write('a.dat') 
!
\end{codeb}
%% --------------------------------------------------------------------------------
\topic{Namelists: fine for derived types, but no pointers or allocatable.}
\begin{codefull}
    type T_RandomVar ! No pointers or allocatables
        character(len=56)   :: sname = ''
        real(MK)            :: value = 0._MK
        type(T_Stats)       :: stats ! No pointers or allocatables
    end type
    type T_Stats ! No pointers or allocatables
        real(MK), dimension(4)  :: moments = (/0._MK,0._MK,0._MK,0._MK/)
    end type
    ![...]
    type(T_RandomVar) :: RandomVar
    ![...]
    namelist/RandomVarInputs/RandomVar
    read(iunit,RandomVarInputs,iostat=ierr)
\end{codefull} 
% --------------------------------------------------------------------------------

% --------------------------------------------------------------------------------
% ---  Precision
% --------------------------------------------------------------------------------
\subsection{Precision}
\label{sec:precision}

\topic{Precision: in general, use a custom module}
\begin{codea}

real*8
integer(int_ptr_kind())
\end{codea}
\begin{codeb}
use SupportPrecision, only: MK, PK
real(MK)
integer(PK)

\end{codeb}
\reason{The syntax \texttt{*8} is depreciated. It is convenient if you need a real that takes exactly 8 bytes, but still, it's depreciated (see next paragraph). Note that \cmd{real*8} and \cmd{real(8)} have no reason to be the same (the kinds are compiler dependent). \cmd{int_ptr_kind} is convenient to support multiple architecture (32/64bits) but is not standard (hence the SupportPrecision module).}
% --------------------------------------------------------------------------------
\topic{Precision: If you really want to precise the size in bytes (8 bit)}
\begin{codea}

real*4, real*8, 
integer*4, integer*8
\end{codea} 
\begin{codeb}
use iso_fortran_env
real(REAL32), real(REAL64)
integer(INT32), real(INT64)
\end{codeb}
\reason{The syntax \texttt{*4} is depreciated. The \cmd{iso_fortran_env} module is not available on old compilers => Use a SupportCompiler module wrapped in a SupportPrecision module. See \cmd{_support/SupportPrecision.f90} \autoref{fil:precision}} 
% --------------------------------------------------------------------------------
\topic{Precision: If you need to communicate with C (recommended for DLLs)}
\begin{codea}

real
double precision
integer
character
logical
\end{codea}
\begin{codeb}
use iso_c_binding
real(C_FLOAT)
real(C_DOUBLE)
integer(C_INT)
character(kind=C_CHAR)
logical(C_BOOL)
\end{codeb}
\reason{The \cmd{iso_c_binding} module is not available on old compilers => Use a SupportCompiler module wrapped in a SupportPrecision module. See \cmd{_support/SupportPrecision.f90} \autoref{fil:precision}} 
% --------------------------------------------------------------------------------
\topic{Precision: use explicit type conversions (with compiler warnings)}
\begin{codea}
!
real*4           :: x
double precision :: y
![...]
y = x  !implicit type conversion
\end{codea} 
\begin{codeb}
use SupportPrecision, only: MPI_DOUBLE, MK
real(MK)         :: x
real(MPI_DOUBLE) :: y
![...]
y = real(x, MPI_DOUBLE) ! explicit conversion 
\end{codeb}
% % 
% \topic{Summary}
% \includecode{_modules_parts/SupportPrecision.f90}

% --------------------------------------------------------------------------------
% --- OS
% --------------------------------------------------------------------------------
\subsection{Operating System}
\label{sec:os}
A lot of fortran builtin routines are cross-platform. 
The main problems can be found when creating directories and inquiring about files.
Compilers have some non-standards extensions. Cross platform solutions can easily be implemented.
The solution advised here is to put OS-specific parameters (like commands, and slash) in a module \cmd{SupportSystem} (see folder \cmd{_support}) which is then included by a \cmd{FileSystem} module (in folder \cmd{_tools}).


% --------------------------------------------------------------------------------
% --- Makefile
% --------------------------------------------------------------------------------
\subsection{Makefile}
\label{sec:makefile}
Makefiles are convenient to compile code on multiple platforms using different compilers and libraries. The current repository contains examples in the \cmd{_\mkf} folder.
\cmd{MakefileOS.mk} attempts to detect the OS and architecure and unify OS-specific parameters.
\cmd{MakefileFortran.mk} attempts to unify Fortan compiler flags.

% --------------------------------------------------------------------------------
% --- Preprocessor 
% --------------------------------------------------------------------------------
\subsection{Preprocessor}
\label{sec:preprocessor}
This section presents some generalities about preprocessors and examples of cases where the preprocessor directives can be replaced by wrapped code in separate modules.
Examples where this approach is not possible are found in \autoref{sec:dlls}.
As mentioned in the introduction, relying on a preprocessor is not really portable and the wrapped approach should be preferred whenever possible. Also, the C-preprocessor being the most used one, it's best to use this one.

\topic{Macros}
The most used feature is something of the form \cmd{if defined MACRO then ...  endif}.
The string \cmd{MACRO} is defined by the compiler or the user.
Macros are defined on the command line using \cmd{-DMACRO}. Since it is a compiler variable intended to be defined in the entire scope of the program, a convention is to surround the macro name with double underscores, e.g. \cmd{__dtu__} (see POSIX standard and ANSI-C standards).
For a list of predefined macros for Compilers/OS/Archi:~\url{http://sourceforge.net/p/predef/wiki/Home/}\\
Given the variability of definitions, it is advised to always (re)define the macros that your are using on the command line: e.g. \cmd{-D__linux__}, \cmd{-D_WIN32}, \cmd{D__intel__}, \cmd{D__compaq__},\cmd{D__amd64__},\cmd{D__i386__}. For instance \cmd{-D__linux__} is not defined by gfortran on linux.

\topic{C-Preprocessor}
The C-Preprocessor is supported by: intel, gfortran, compaq, sun, pgi portland.\\
Macros are \textbf{case sensitive}.

gfortran: 
\cmd{-cpp}: use C-preprocessor,
\cmd{-E -dM}: show preprocessor macros\\
ifort :
\cmd{-fpp}: use C-preprocessor,
\cmd{-E -dM -dryrun}: show preprocessor macros\\
sun :
\cmd{-xpp}: use C-preprocessor\\
pgf90:
\cmd{-Mpreproc} : use C-preprocessor\\
%
\topic{DEC-Preprocessor}
The DEC-Preprocessor is supported by: intel and compaq.\\
Macros are \textbf{case in-sensitive}.

\topic{GNU-Preprocessor}
The GNU-Preprocessor is supported by: gfortran.\\
It is only used to define \cmd{ATTRIBUTES}, it doesn't support \cmd{if defined}.\\
\cmd{!GCC$ ATTRIBUTES DLLEXPORT :: init}



\topic{Preprocessor: Examples where preprocessor directives can be removed}
\begin{codea}
!DEC$ IF DEFINED(__HDF5__)
   call hdf5_init()
!DEC$ END IF
\end{codea}
\begin{codeb}
use SupportHDF5, only: hdf5_init()
call hdf5_init()
!
\end{codeb}\\
\begin{codea}
! C preprocessor
#if defined _WIN32
  call mkdir_windows('fold')
#elif defined __unix__
  call mkdir_linux('fold')
#endif
\end{codea}
\begin{codeb}
! FileSystem is found in _tools, it uses SupportSystem
use FileSystem, only: system_mkdir 

call system_mkdir('fold')
!
\end{codeb}
A possible variation (not as clean):\\
\begin{codea}
! C preprocessor
#if defined _WIN32
  !do something
#elif defined __unix__
  !do another thing
#endif
\end{codea}
\begin{codeb}
use SupportSystem, only: OSNAME
if (OSNAME(1:7)=='windows') then
    !do something
elseif if (OSNAME(1:5)=='linux') then
    !do another thing
endif
\end{codeb}
\reason{The if statements will not affect performances since they relies on compile time constants. Compiler optimization should remove dead-code and dead-if statements. This method cannot be used around ``use'' statements or routines declarations.}


\subsection{DLLs, cross-language interoperability}
\label{sec:dlls}
-For C-strings see \autoref{sec:characters} and the file \verb|CStrings.f90|\\
-For C-types see \autoref{sec:precision}\\

\topic{Procedure names/alias: bind(C) is really convenient, but not supported by Compaq}
\begin{codea}
subroutine init(array1)
!DEC$ ATTRIBUTES C, ALIAS:'init'::init
\end{codea}
\begin{codeb}
subroutine init(array1) bind(C,name='init')
!
\end{codeb}
\reason{The code on the left is standard 2003, cross-platform, cross-compiler, preprocessor-independent and just easy to use. The only down side is that the Compaq compiler does not support it. Note: for dllexport it makes it easier if the subroutine name and the bind-name are the same.
    NOTE: An array dummy argument of a BIND(C) procedure must be an explicit shape (\cmd{dimension(n), dimension(n,m)}) or assumed size array (\cmd{dimension(*), dimension(lda,*)}).
    If it's an assumed size array, the size of the array is not computable and thus the upper bound should always be precised, i.e. \cmd{A(1,:)} should be something like \cmd{A(1,1:n)} } 



\topic{Procedure exports for dll: the problem of the def file}
\begin{codea}
subroutine init(array1) bind(c,name='init')
!DEC$ IF .NOT. DEFINED(__LINUX__)
!DEC$ ATTRIBUTES DLLEXPORT ::init
!GCC$ ATTRIBUTES DLLEXPORT ::init
!DEC$ END IF
\end{codea}
\begin{codeb}
subroutine init(array1) bind(C,name='init')


 
! Generate the def file yourself
\end{codeb}
\reason{The code above is not compatible with old compilers like Compaq due to the \texttt{bind(C)} directive. The code on the left should work for Intel and GCC but it relies on preprocessor directives.
The code on the right is clean and portable. It requires more work on the windows users since the \texttt{.def} file needs to be written. A dll interface is not expected to change that often, so the work is not that heavy. The python tool \texttt{simple-fortran-parser} can generate the \texttt{.def} automatically based on all the \texttt{bind(C)} subroutines it finds in the code.}



\topic{Procedure exports for dll: a more-or-less portable way}
\begin{codea}
#if defined OLD_COMPILER
subroutine init(array1)
!DEC$ ATTRIBUTES C, ALIAS:'init'::init
#else
subroutine init(array1) bind(c,name='init')
#endif
!DEC$ IF .NOT. DEFINED(__LINUX__)
!DEC$ ATTRIBUTES DLLEXPORT ::init
!GCC$ ATTRIBUTES DLLEXPORT ::init
!DEC$ END IF
\end{codea}
\begin{codeb}
#if defined OLD_COMPILER
subroutine init(array1)
!DEC$ ATTRIBUTES C, ALIAS:'init'::init
#else
subroutine init(array1) bind(c,name='init')
#endif
!
!
! Generate the def file yourself
!
\end{codeb}
\reason{The above should work with compaq,intel,gcc on windows and linux as long a C preprocessor flag is given to the compilers (i.e. \texttt{-fpp} or \texttt{-cpp}) and as long as the Compaq compiler defines the flag \texttt{-DOLD\_COMPILER}. On linux with gfortran the \texttt{-Wno-attributes} could be use to avoid the warning. }












% --------------------------------------------------------------------------------
% ---  
% --------------------------------------------------------------------------------
\section{Support files}
% \topic{Summary}
\subsection{Compiler (example for intel)}
\label{fil:compiler}
\includecode{../_support/SupportCompiler_intel.f90}



\subsection{Precision}
\label{fil:precision}
\includecode{../_support/SupportPrecision.f90}


\subsection{Sytem (example for linux)}
\label{fil:system}
\includecode{../_support/SupportSystem_linux.f90}


\subsection{C Strings}
\label{fil:cstrings}
\includecode{../_tools/CStrings.f90}


\end{document}
