\documentclass{report} 
\usepackage[T1]{fontenc}
\usepackage{lmodern}
\usepackage{amsmath}
\usepackage{amssymb}
\usepackage{listings} % a inclure pour la fonction listing
\usepackage{longtable}
\usepackage{geometry}
\usepackage{ifpdf}
\ifpdf
	\usepackage{graphicx}
	\usepackage{epstopdf} %don't forget the shell-escape flag for pdflatex
\else
	\usepackage{graphicx}
\fi
\usepackage{hyperref}
\usepackage{color} 
\definecolor{mygreen}{rgb}{0,0.6,0}
\definecolor{gret}{rgb}{0,0.6,0}
\definecolor{mygray}{rgb}{0.5,0.5,0.5}
\definecolor{mymauve}{rgb}{0.58,0,0.82}
\definecolor{orange}{rgb}{1,0.5,0}
\definecolor{grey}{rgb}{0.88,0.88,0.88}
\definecolor{midgrey}{rgb}{0.5,0.5,0.50}
\definecolor{greylight}{rgb}{0.95,0.95,0.95}
\definecolor{greensoft}{rgb}{0.25,0.95,0.25}
%%for dtu
\definecolor{dtured}{rgb}{0.6706,0.20784,0.227451}
\definecolor{dtugrey}{rgb}{0.5,0.5,0.5}


% --------------------------------------------------------------------------------
% --- Code 
% --------------------------------------------------------------------------------
\newcommand{\basiclstset}{%
\lstset{ %
  backgroundcolor=\color{white},   % choose the background color; 
  basicstyle=\footnotesize,        % the size of the fonts that are used for the code
  breakatwhitespace=false,         % sets if automatic breaks should only happen at whitespace
  breaklines=true,                 % sets automatic line breaking
  commentstyle=\color{mygreen},    % comment style
  deletekeywords={...},            % if you want to delete keywords from the given language
  escapeinside={\%*}{*)},          % if you want to add LaTeX within your code
  extendedchars=true,              % lets you use non-ASCII characters; for 8-bits encodings only, does not work with UTF-8
%   frame=single,	                   % adds a frame around the code
%   keepspaces=true,                 % keeps spaces in text, useful for keeping indentation of code (possibly needs columns=flexible)
  keywordstyle=\color{blue},       % keyword style
  language=Fortran,                 % the language of the code
  otherkeywords={},            % if you want to add more keywords to the set
%   numbersep=5pt,                   % how far the line-numbers are from the code
%   numberstyle=\tiny\color{mygray}, % the style that is used for the line-numbers
  rulecolor=\color{black},         % if not set, the frame-color may be changed on line-breaks within not-black text (e.g. comments (green here))
  showspaces=false,                % show spaces everywhere adding particular underscores; it overrides 'showstringspaces'
  showstringspaces=false,          % underline spaces within strings only
  showtabs=false,                  % show tabs within strings adding particular underscores
%   stepnumber=2,                    % the step between two line-numbers. If it's 1, each line will be numbered
  stringstyle=\color{mymauve},     % string literal style
%   tabsize=2,	                   % sets default tabsize to 2 spaces
%   title=\lstname,                   % show the filename of files included with \lstinputlisting; also try caption instead of title
%   belowskip=-2.0 \baselineskip,
%   belowcaptionskip=0cm,
%   aboveskip=0cm,
%    belowskip=3em,
% linewidth=\linewidth
}
}


% --------------------------------------------------------------------------------
% ---  
% --------------------------------------------------------------------------------
\lstloadlanguages{Fortran}
\lstnewenvironment{code}
{\minipage{0.49\textwidth}}
    {\endminipage}
\lstnewenvironment{codefull}
{\minipage{1.0\textwidth}}
    {\endminipage}

\lstnewenvironment{good}
{\hfill\minipage{0.47\textwidth}\lstset{title=\textbf{good}}}
    {\endminipage\hfill\null}
\lstnewenvironment{bad}
{\hfill\minipage{0.47\textwidth}\lstset{title=\textbf{bad}}}
    {\endminipage\hfill\null}
% --------------------------------------------------------------------------------
% --- Personal commands 
% --------------------------------------------------------------------------------
\newcommand{\weird}[1]{\textcolor{red}{!!!#1 }}

\newcommand{\topic}[1]{%
   \multicolumn{2}{p{\textwidth}}{\newline\textbf{#1}}\\
}
\newcommand{\reason}[1]{%
    \multicolumn{2}{p{\textwidth}}{#1}\\
}

% --------------------------------------------------------------------------------
% --- Document 
% --------------------------------------------------------------------------------
\begin{document}
\basiclstset

% --------------------------------------------------------------------------------
% ---  Precision
% --------------------------------------------------------------------------------
\section*{Precision}
% --------------------------------------------------------------------------------
% ---  
% --------------------------------------------------------------------------------
\begin{longtable}{p{0.5\linewidth}p{0.5\linewidth}}
% \hline
%     \textbf{Don't use}& \textbf{Use}\\
% --------------------------------------------------------------------------------
\topic{Precision: in general, use a custom module}
\begin{code}

    real*4, real*8, 
    integer(int_ptr_kind())
\end{code} &
\begin{code}
    use SupportPrecision
    real(SP), real(DP)
    integer(PTRK)

\end{code}\\
% --------------------------------------------------------------------------------
\topic{Precision: If you really want to precise the size in bytes (8 bit)}
\begin{code}

    real*4, real*8, 
    integer*4, integer*8
\end{code} &
\begin{code}
    use iso_fortran_env
    real(REAL32), real(REAL64)
    integer(INT32), real(INT64)
\end{code}\\
\reason{Note: the syntax is non-standard and depreciated. The iso-c-binding module is FORTRAN 2003 and not available on old compilers => Use a SupportCompiler module} 
% --------------------------------------------------------------------------------
\topic{Precision: If you need to communicate with C (recommended for DLLs)}
\begin{code}

    real
    double precision
    integer
    character
    logical
\end{code} &
\begin{code}
    use iso_c_binding
    real(C_FLOAT)
    real(C_DOUBLE)
    integer(C_INT)
    character(kind=C_CHAR)
    logical(C_BOOL)
\end{code}\\
\reason{Note: the syntax is non-standard and depreciated. The iso-fortran-env module is not available on old compilers => Use a SupportCompiler module} 
% --------------------------------------------------------------------------------
\end{longtable}


Summary\\
\begin{codefull}
module SupportPrecision
  ! Compiler interface to iso_c_binding
  use SupportCompiler, only: C_FLOAT, C_DOUBLE, C_CHAR, C_INT, C_BOOL
  ! Compiler interface to iso_fortran_env
  use SupportCompiler, only: REAL32, REAL64, INT32, INT64
  ! Compiler interface to int_ptr_kind
  use SupportCompiler, only: IPTRK
  !
  integer, parameter :: SP = REAL32
  integer, parameter :: DP = REAL64
  integer, parameter :: MK = C_DOUBLE
end module
\end{codefull}







\end{document}
